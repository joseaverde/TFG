\chapter{Implementación y despliegue}\label{chap:implementation}
\section{Definición del problema}
Un señal de encefalograma $s\in\mathcal{S}=\bigcup_{k=0}^\infty \mathbb{R}^k$
es una secuencia potencialmente infinita de muestras $x\in\mathbb{R}$ de un
sensor que está conectado a la cabeza y muestrea continuamente las señales que
recibe del cerebro a un ritmo constante. Se denota la $k$-ésima muestra de una
señal como $s (k)\in\mathbb{R}$, donde $k=1,2,3,...$. Una porción o un
fragmento es un secuencia continua de muestras dentro una señal dada, la señal
desde la $k$-ésima hasta la $i$-ésima muestra (ambas incluidas) se denota como
$s (k .. i)\in\mathcal{S}$.

El objetivo es determinar si cierta porción de una señal $s(k..k+e)$ tiene un
ataque epiléptico a partir de las características (\textit{features}) propias
de la señal y de cada paciente individual. Para ello se entrena un modelo que
trata de maximizar el número de verdaderos positivos y negativos; y minimizar
el número de falsos positivos y negativos. La función
$\mathscr{S}: \mathcal{S} \rightarrow \mathcal{M} \rightarrow \mathbb{B}$
retorna si un fragmento de la señal $s\in\mathcal{S}$, para un modelo dado
$m\in\mathcal{M}$ es o no un ataque epiléptico ($\mathbb{B}=\{\top, \bot\}$, es
el conjunto de los valores lógico cierto [$\top$] y falso [$\bot$]).

Previamente, para cada paciente se entrena un modelo a partir de las
características de la señal de encefalograma durante un periodo de tiempo en
que ha habido varios ataques epilépticos. Estas características son siete:

\begin{itemize}
    \item $PSD_1$: Rango de densidad espectral de potencia entre 2 Hz y 12 Hz.
    \item $PSD_2$: Rango de densidad espectral de potencia entre 12 Hz y 18 Hz.
    \item $PSD_3$: Rango de densidad espectral de potencia entre 18 Hz y 35 Hz.
    \item $E$: Rango de energía de la señal.
    \item $D_M$: Rango de la distancia máxima entre la muestra mayor y menor.
    \item $d_{max,C}$: Distancia máxima entre un fragmento de señal dado y cada
        uno de los patrones $P_j$ utilizando el algoritmo de deformación
        dinámica del tiempo. Es decir si el resultado de dicho algoritmo se
        encuentra en $[0, d_{max,C}]$, se considera un ataque.
    \item $P=P_j$: es una lista de patrones $P_j\in\mathcal{S}$.
\end{itemize}

El modelo es la siete-tupla que contiene los cinco rangos, $d_{max,C}$ y los
$n$ patrones $P_j$:

    \[ \mathcal{M} = (PSD_1, PSD_2, PSD_3, E, D_m, d_{max,c}, P) \]

Se dice que se está dentro de una región de ataque epiléptico $s\in\mathbb{S}$
para un modelo $m\in\mathcal{M}$ si:

\begin{align*}
    \mathscr{S}(s,m) &  =    PSD(s, f,  2,12)\in[PSD_{1_{min}},PSD_{1_{max}}]\\
                     &~\land PSD(s, f, 12,18)\in[PSD_{2_{min}},PSD_{2_{max}}]\\
                     &~\land PSD(s, f, 18,35)\in[PSD_{3_{min}},PSD_{3_{max}}]\\
                     &~\land Energy(s)       \in[E_{min}, E_{max}]        \\
                     &~\land Max\_Dist(s)    \in[D_{M_{min}}, D_{M_{max}}]\\
                     &~\land \exists P_j\in P: DTW(s, P_j) \in [0, d_{max,c}]\\
\end{align*}


\section{Verificación Formal}
De acuerdo con el estudio, para determinar si una época
\subsection{Distancia Máxima}
O en inglés \textit{max distance} es uno de los parámetros para filtrar un
\textit{epoch}

\begin{algorithm}[h]
    \caption{Distancia Máxima}
    \KwData{$S$}
    \KwResult{$M - m$}

    \SetAlgoLined
    $m \longleftarrow S(1)$\;
    $M \longleftarrow S(1)$\;

    \For{$I\in2 .. N$}{
        \If{$S(I) < m$}{
            $m \longleftarrow S(I)$\;
        }
        \If{$S(I) > M$}{
            $M \longleftarrow S(I)$\;
        }
    }
    \Return{$M - m$}
\end{algorithm}

\subsection{Energía}


